\documentclass[11pt]{article}

\usepackage{amsmath}
\usepackage{amsthm}
\usepackage{slashed}

\title{P\&S QFT - Problem 5.1 / part 2}
\author{Fedor Indutny}

\begin{document}

\maketitle

Work out cross section for electron-muon scattering, in the muon rest frame,
retaining the electron mass but sending $m_\mu \to \infty$.

\section{Expected result}

\begin{equation}
\frac{d \sigma}{d \Omega} = \frac{{\cal{Z}}^2 \alpha^2}{4 |\textbf{p}|^2
  \beta^2 \sin^4(\theta/2)}
\Big(1 - \beta^2 \sin^2 \frac{\theta} {2}\Big)
\end{equation}

\section{Definitions}

\begin{itemize}
  \item $E$ - energy of electron
  \item $\beta$ - speed of electron
  \item $p$, $p'$ - 4-momentum of ingoing/outgoing electrons
  \item $k$, $k'$ - 4-momentum of ingoing/outgoing muon
  \item $t = (p - p')^2$ - Photon momentum (Mandelstam variable)
\end{itemize}

\section{Solution}

Scattering amplitude \textit{(spinor indices omitted)}:

\begin{equation}
  % M
  {\cal{M}} = \overline{u}(p') (-i e \gamma^\mu) u(p)
  \Big(\frac{-i g_{\mu \nu}}{t + i \varepsilon} \Big)
  \overline{u}(k') (-i e \gamma^\nu) u(k)
\end{equation}

Averaging $|{\cal{M}}|^2$ over all spin configurations and dividing by $4$
(for input spins):
\begin{equation}
  \begin{split}
    % M^2
    {|\cal{M}|}^{2} = \frac{e^4}{4 t^2}
    % Electron trace
    & tr \Big[ (\slashed{p'} + m) \gamma^\mu (\slashed{p} + m) \gamma^\nu \Big]
    % Muon trace
    \cdot \\
    & tr \Big[
      (\slashed{k'} + m_\mu) \gamma_\mu (\slashed{k} + m_\mu) \gamma_\nu  \Big]
  \end{split}
\end{equation}

Working out traces separately:
\begin{equation}
  \begin{split}
    tr & \Big[ (\slashed{p'} + m) \gamma^\mu
      (\slashed{p} + m) \gamma^\nu \Big] =
    tr \Big[ \slashed{p'} \gamma^\mu \slashed{p} \gamma^\nu \Big] +
    m^2 tr \Big[ \gamma^\mu \gamma^\nu \Big] \\
    %
    % Second line
    %
    & = p'^\rho p^\sigma tr \Big [
      \gamma^\rho \gamma^\mu \gamma^\sigma \gamma^\nu \Big ] +
    4 m^2 g^{\mu\nu} \\
    %
    % Third line
    %
    & = 4 p'^\rho p^\sigma \Big ( g^{\rho \mu} g^{\sigma \nu} - 
      g^{\rho \sigma} g^{\mu \nu} + g^{\rho \nu} g^{\mu \sigma} \Big ) +
      4 m^2 g^{\mu \nu} \\
    %
    % Last line
    %
    & = 4 \Big ( (m^2 - p \cdot p') g^{\mu \nu} +
      (p^\mu p'^\nu + p^\nu p'^\mu) \Big )
  \end{split}
\end{equation}

Other trace:
\begin{equation}
    tr \Big[ (\slashed{k'} + m_\mu) \gamma_\mu
      (\slashed{k} + m_\mu) \gamma_\nu \Big] =
      4 \Big ( (m_\mu^2 - k \cdot k') g_{\mu \nu} +
                   (k_\mu k'_\nu + k_\nu k'_\mu) \Big )
\end{equation}

We're going to split ${|\cal{M}|}^2$ into four parts:
\begin{equation}
  {|\cal{M}|}^2 = \frac{e^4}{4t^2} \Big [ (I) + (II) + (III) + (IV) \Big ]
\end{equation}

Doing $(I)$:

\begin{equation}
  \begin{split}
    (I) & = 16 (m^2 - p \cdot p') g^{\mu \nu} (m_\mu^2 - k \cdot k')
      g_{\mu \nu} \ \\
    & = 64 (m^2 - p \cdot p') (m_\mu^2 - k \cdot k')
  \end{split}
\end{equation}

Doing $(II)$:

\begin{equation}
  \begin{split}
    (II) & = 16 (p^\mu p'^\nu + p^\nu p'^\mu) (m_\mu^2 - k \cdot k')
      g_{\mu \nu} \\
    & = 32 p \cdot p' (m_\mu^2 - k \cdot k')
  \end{split}
\end{equation}

Naturally $(III)$ is:

\begin{equation}
  \begin{split}
    (III) = 32 k \cdot k' (m^2 - p \cdot p')
  \end{split}
\end{equation}

$(IV)$:

\begin{equation}
  \begin{split}
    (IV) & = 16 (p^\mu p'^\nu + p^\nu p'^\mu) (k_\mu k'_\nu + k_\nu k'_\mu) \\
    & = 32 (p \cdot k p' \cdot k' + p \cdot k' p' \cdot k)
  \end{split}
\end{equation}

Summing all together:
\begin{equation}
  \begin{split}
    % First line
    {|\cal{M}|}^2 & = \frac{8 e^4}{t^2} \Big [
      2 (m^2 - p \cdot p') (m_\mu^2 - k \cdot k') +
      p \cdot p' (m_\mu^2 - k \cdot k') + \\
      & \phantom{=====} k \cdot k' (m^2 - p \cdot p') +
      p \cdot k p' \cdot k' + p \cdot k' p' \cdot k \Big ] \\
    % Second line
    & = \frac{8 e^4}{t^2} \Big [
      2 m^2 m_\mu^2 - 2 m^2 k \cdot k' - 2 m_\mu^2 p \cdot p' +
        2 p \cdot p' k \cdot k' + \\
      & \phantom{=====} m_\mu^2 p \cdot p' - p \cdot p' k \cdot k' +
      m^2 k \cdot k' - p \cdot p' k \cdot k' + \\
      & \phantom{=====} p \cdot k p' \cdot k' + p \cdot k' p' \cdot k \Big ] \\
    % Third line
    & = \frac{8 e^4}{t^2} \Big [
      2 m^2 m_\mu^2 - m^2 k \cdot k' - m_\mu^2 p \cdot p' +
      p \cdot k p' \cdot k' + p \cdot k' p' \cdot k \Big ]
  \end{split}
\end{equation}

Specialize $p$, $k$, $p'$, $k'$:
\begin{equation}
  \begin{split}
    p & = (E,0,0,|\textbf{p}|) \\
    k & = (m_\mu, 0, 0, 0) \\
    p' & = (E, 0, 0, \sin \theta |\vec{p'}|, \cos \theta |\vec{p'}|) \\
    k' & = (m_\mu, 0, -\sin \theta |\vec{p'}|, |\textbf{p}| -
      \cos \theta |\vec{p'}|)
  \end{split}
\end{equation}

Time to do differential cross section ($u = |\vec{p'}|$):
\begin{equation}
  \begin{split}
  d\sigma & = \frac{1}{2 E 2 m_\mu \beta}
    \frac{d^3 p'}{(2 \pi)^3} \frac{1} {2 E'}
    \frac{d^3 k'} {(2 \pi)^3} \frac{1} {2 m_\mu}
    |{\cal{M}}|^2 (2 \pi)^4 \delta^{(4)}(p + k - p' - k')
  \end{split}
\end{equation}

We can use 3 of 4 delta functions to eliminate $d^3k'$:  
  
\begin{equation}
  \begin{split}
  \ldots & = \frac{1}{4 (4 \pi)^2 E m_\mu^2 \beta} \frac{d^3p'} {E'}
    |{\cal{M}}|^2 \delta(E - E') \\
  & = \frac{1}{4 (4 \pi)^2 E m_\mu^2 \beta} \frac{d\Omega \, du}{E'}
    u^2 |{\cal{M}}|^2 \delta(E - E')
  \end{split}
\end{equation}

When in doubt - integrate! \textit{(and use $E' = \sqrt{m^2 + u^2}$)}:

\begin{equation}
  \int_0^\infty \frac{du} {E'} u^2 |{\cal{M}}|^2 \delta(E - E')
  = \int_0^\infty dE' u |{\cal{M}}|^2 \delta(E - E')
  = |\textbf{p}| |{\cal{M}}|^2
\end{equation}

Thus:
\begin{equation}
  \frac{d\sigma}{d\Omega} = \frac{1}{4 (4 \pi)^2 E m_\mu^2 \beta}
    |\textbf{p}| |{\cal{M}}|^2
\end{equation}

Given that $E\beta = |\textbf{p}|$:

\begin{equation}
  \frac{d\sigma}{d\Omega} = \frac{|{\cal{M}}|^2}{4 (4 \pi)^2 m_\mu^2}
\end{equation}

Few useful dot products:
\begin{equation}
  \begin{split}
    p \cdot k & = E m_\mu \\
    p \cdot k' &= E m_\mu - |\textbf{p}|^2 (1 - \cos \theta) \\
    p' \cdot k &= E m_\mu \\
    p' \cdot k' &=  E m_\mu + \sin^2 \theta |\textbf{p}|^2 +
      \cos^2 \theta |\textbf{p}|^2 -
      \cos \theta |\textbf{p}|^2 \\
      &= E m_\mu + |\textbf{p}|^2 (1 - \cos \theta)\\
    p \cdot p' & = E^2 - \cos \theta |\textbf{p}| u \\
    k \cdot k' & = m_\mu^2 \\
    t & = (p - p')^2 = -\sin^2 \theta |\textbf{p}|^2 -
      (\cos \theta |\textbf{p}| - |\textbf{p}|)^2 \\
      & = -2 |\textbf{p}|^2 (1 - \cos \theta) \\
      & = -4 |\textbf{p}|^2 \sin^2 \frac{\theta} {2} \\
    t^2 & = 16 |\textbf{p}|^4 \sin^4 \frac{\theta} {2}
  \end{split}
\end{equation}

Let's put some things together:
\begin{equation}
  \begin{split}
    |{\cal{M}}|^2 & = \frac{8 e^4}{t^2} \Big [
      2 m^2 m_\mu^2 -
      % m^2 k \cdot k'
      m^2 m_\mu^2 -
      % m_\mu^2 p \cdot p'
      m_\mu^2 (E^2 - \cos \theta |\textbf{p}|^2) + \\
      % p \cdot k p' \cdot k'
      & \phantom{=====} E m_\mu
        ( E m_\mu + |\textbf{p}|^2 (1 - \cos \theta) \big ) + \\
      % p \cdot k' p' \cdot k
      & \phantom{=====} \big ( E m_\mu -
        |\textbf{p}|^2 (1 - \cos \theta) \big ) E m_\mu
      \Big ] \\
    & = \frac{8 e^4}{t^2} \Big [
      m^2 m_\mu^2 + \cos \theta |\textbf{p}|^2 m_\mu^2 + E^2 m_\mu^2 \Big ]
  \end{split}
\end{equation}

\begin{equation}
  \begin{split}
    \frac{|{\cal{M}}|^2} {m_\mu^2} & = \frac{8 e^4}{t^2} \Big [
      m^2 + (1 - 2 \sin^2 \frac{\theta} {2}) |\textbf{p}|^2 + E^2 \Big ] \\
    & = \frac{16 e^4}{t^2} \Big ( E^2 - \sin^2 \frac {\theta} {2}
      |\textbf{p}|^2 \Big ) \\
    & = \frac{e^4}{|\textbf{p}|^4 \sin^4 \frac{\theta} {2}}
      \Big ( E^2 - \sin^2 \frac {\theta} {2} |\textbf{p}|^2 \Big ) \\
    & = \frac{e^4}{|\textbf{p}|^4 \sin^4 \frac{\theta} {2}}
      E^2 \Big ( 1 - \sin^2 \frac {\theta} {2} \beta^2 \Big ) \\
    & = \frac{e^4}{|\textbf{p}|^2 \beta^2 \sin^4 \frac{\theta} {2}}
      \Big ( 1 - \sin^2 \frac {\theta} {2} \beta^2 \Big )
  \end{split}
\end{equation}

And final:

\begin{equation}
  \begin{split}
    \frac{d\sigma}{d\Omega} & = \frac{|{\cal{M}}|^2} {4 (4 \pi)^2 m_\mu^2} \\
    & = \frac{\alpha^2} {4 |\textbf{p}|^2 \beta^2 \sin^4 \frac{\theta} {2}}
      \Big ( 1 - \beta^2 \sin^2 \frac {\theta} {2} \Big ) \implies \qed
  \end{split}
\end{equation}

\end{document}