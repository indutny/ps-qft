\documentclass[11pt]{article}

\usepackage{amsmath}
\usepackage{amsthm}
\usepackage{amsfonts}
\usepackage{slashed}

\title{P\&S QFT - Chapter 3 problems}
\author{Fedor Indutny}

\begin{document}

\maketitle

\section*{3.1}

\begin{equation}
[ J^{\mu\nu}, J^{\rho\sigma} ] = i \left(
  g^{\nu\rho} J^{\mu\sigma} - g^{\mu\rho} J^{\nu\sigma} -
  g^{\nu\sigma} J^{\mu\rho} + g^{\mu\sigma} J^{\nu\rho}
\right)
\end{equation}

\paragraph*{(a)}

\begin{center}
$L^i = \frac{1}{2} \epsilon^{ijk} J^{jk}$, $K^i = J^{0i}$
\end{center}

\begin{equation}
\begin{split}
[ L^i, L^j ] & =
  \frac{1}{4} \epsilon^{ikl} \epsilon^{jmn} [ J^{kl}, J^{mn} ] \\
  & = -\frac{i}{4} \epsilon^{ikl} \epsilon^{jmn} \left(
    \delta^{lm} J^{kn} - \delta^{km} J^{ln} -
    \delta^{ln} J^{km} + \delta^{kn} J^{lm}
  \right) \\
  & = \textit{(...renaming summation variables...)} \\
  & = -\frac{i}{4} \left(
    \epsilon^{ikl} \epsilon^{jln} -
    \epsilon^{ilk} \epsilon^{jln} -
    \epsilon^{ikl} \epsilon^{jnl} +
    \epsilon^{ilk} \epsilon^{jnl}
  \right) J^{kn} \\
  & = -i \epsilon^{ikl} \epsilon^{jln} J^{kn} =
  -i \epsilon^{lik} \epsilon^{lnj} J^{kn} =
  -i \left( \delta^{in} \delta^{kj} - \delta^{ij} \delta^{kn} \right)
    J^{kn} \\
  & = -i \left( J^{ji} - \delta^{ij} J^{kk} \right) =
    i J^{ij}
\end{split}
\end{equation}

Now:

\begin{equation}
\epsilon^{ijk} L^k =
  \frac{1}{2} \epsilon^{ijk} \epsilon^{klm} J^{lm} =
  \frac{1}{2} \left(
    \delta^{il} \delta^{jm} - \delta^{im} \delta^{lj}
  \right) J^{lm} = J^{ij}
\end{equation}

Thus:

\begin{equation}
[ L^i, L^j ] = i \epsilon^{ijk} L^k
\end{equation}

\pagebreak

Commutator of boost and rotation:

\begin{equation}
\begin{split}
[ L^i, K^j ] & = \frac{1}{2} \epsilon^{ikl} [ J^{kl}, J^{0j} ] =
  \frac{i}{2} \epsilon^{ikl} \left(
    g^{l0} J^{kj} - g^{k0} J^{lj} - g^{lj} J^{k0} + g^{kj} J^{l0}
  \right) \\
& = \frac{i}{2} \epsilon^{ikl}
  \left( g^{lj} K^k - g^{kj} K^l \right) =
  -\frac{i}{2}
  \left( \epsilon^{ikj} K^k - \epsilon^{ijl} K^l \right) \\
& = i \epsilon^{ijk} K^k
\end{split}
\end{equation}

Commutator of two boosts:

\begin{equation}
[ K^i, K^j ] = [ J^{0i}, J^{0j} ] =
-i J^{ij} = - i \epsilon^{ijk} L^k
\end{equation}

Combinations:

\begin{equation}
\mathbf{J}_+ = \frac{1}{2} \left( \mathbf{L} + i \mathbf{K} \right)
\end{equation}

\begin{equation}
\mathbf{J}_- = \frac{1}{2} \left( \mathbf{L} - i \mathbf{K} \right)
\end{equation}

Commutator of combinations:

\begin{equation}
\begin{split}
[ J_+^i, J_-^j ] & = \frac{1}{4} [ L^i + i K^i, L^j - i K^j ] \\
& = \frac{1}{4} \left( [ L^i, L^j ] + [ K^i, K^j ] +
  i [ K^i, L^j ] + i [ K^j, L^i ] \right) \\
& = \frac{1}{4} \left(
  i \epsilon^{ijk} L^k - i \epsilon^{ijk} L^k +
  i \epsilon^{ijk} K^k - i \epsilon^{ijk} K^k \right) = 0
\end{split}
\end{equation}

Commutator of the components of $J_+$:
\begin{equation}
\begin{split}
[ J_+^i, J_+^j ] & = \frac{1}{4} \left( [ L^i, L^j ] - [ K^i, K^j ] +
  i [ K^i, L^j ] - i [ K^j, L^i ] \right) \\
& = \frac{i}{4} \left(
  \epsilon^{ijk} L^k + \epsilon^{ijk} L^k + i \epsilon^{ijk} K^k +
  i \epsilon^{ijk} K^k
\right) \\
& = i \epsilon^{ijk} J_+^k
\end{split}
\end{equation}

Similar result holds for $\mathbf{J}_-$.

\pagebreak

\paragraph*{(b)} For spin $\frac{1}{2}$ representations
$\mathbf{J} = \frac{\boldsymbol{\sigma}}{2}$. Since $\mathbf{J}_+$ and
$\mathbf{J}_-$ commute the states in the Lorentz group representation
naturally decouple into a tensor product of eigenstates of
$\mathbf{J}_+$ and $\mathbf{J}_-$.

For $(j_+, j_-) = (\frac{1}{2}, 0)$ representation:

\begin{equation}
\mathbf{L} = \mathbf{J}_+ + \mathbf{J}_- =
  \frac{\boldsymbol{\sigma}}{2}
\end{equation}

\begin{equation}
\mathbf{K} = \frac{1}{i} \left( \mathbf{J}_+ - \mathbf{J}_- \right) =
  \frac{\boldsymbol{\sigma}}{2i}
\end{equation}

Thus $(\frac{1}{2}, 0)$ state transforms as:

\begin{equation}
\psi_L \rightarrow
  \left(
    1 - i \boldsymbol{\theta} \cdot \frac{\boldsymbol{\sigma}}{2} -
    \boldsymbol{\beta} \cdot \frac{\boldsymbol{\sigma}}{2}
  \right) \psi_L
\end{equation}

While $(0, \frac{1}{2})$ state transforms as:

\begin{equation}
\psi_R \rightarrow
  \left(
    1 - i \boldsymbol{\theta} \cdot \frac{\boldsymbol{\sigma}}{2} +
    \boldsymbol{\beta} \cdot \frac{\boldsymbol{\sigma}}{2}
  \right) \psi_R
\end{equation}

\paragraph*{(c)}

\begin{equation}
\boldsymbol{\sigma}^T = - \sigma^2 \boldsymbol{\sigma} \sigma^2
\end{equation}

The $(\frac{1}{2}, \frac{1}{2})$ matrix representation is essentially
a dot product of the vector and $\sigma$:
\begin{equation}
\begin{pmatrix}
V^0 + V^3 & V^1 - i V^2 \\
V^1 + i V^2 & V^0 - V^3
\end{pmatrix} = V^0 + \mathbf{V} \cdot \boldsymbol{\sigma} 
\end{equation}

To ease calculations define:
\begin{equation}
\boldsymbol{\tau} \equiv
  \frac{1}{2} \left(
    \boldsymbol{\beta} - i \boldsymbol{\theta} \right)
\end{equation}

Applying right transformation on the left:
\begin{equation}
\begin{split}
\left(
  1 +
  \boldsymbol{\tau} \cdot \boldsymbol{\sigma}
\right) \left( V^0 + \mathbf{V} \cdot \boldsymbol{\sigma} \right) =
\left( V^0 + \boldsymbol{\tau} \cdot \mathbf{V} \right) +
\left( V^0 \boldsymbol{\tau} + \mathbf{V} +
  i \boldsymbol{\tau} \times \mathbf{V} \right) \cdot
  \boldsymbol{\sigma}
\end{split}
\end{equation}

Applying transposed left transformation on the right:

\begin{equation}
\begin{split}
& \left(
  \left( V^0 + \boldsymbol{\tau} \cdot \mathbf{V} \right) +
  \left( V^0 \boldsymbol{\tau} + \mathbf{V} +
  i \boldsymbol{\tau} \times \mathbf{V} \right) \cdot
  \boldsymbol{\sigma}
\right) \left(
  1 + \sigma^2 \boldsymbol{\tau}^* \cdot \boldsymbol{\sigma} \sigma^2 \right) = \\
& \left( V^0 + \boldsymbol{\tau} \cdot \mathbf{V} \right)
\end{split}
\end{equation}

\end{document}