\documentclass[11pt]{article}

\usepackage{amsmath}
\usepackage{amsthm}
\usepackage{slashed}

\title{P\&S QFT - Chapter 2 problems}
\author{Fedor Indutny}

\begin{document}

\maketitle

\section*{2.1}

\begin{equation}
S = \int d^4 x \left( - \frac{1}{4} F_{\mu\nu} F^{\mu\nu} \right)
\end{equation}

\begin{equation}
F_{\mu\nu} = \partial_\mu A_\nu - \partial_\nu A_\mu
\end{equation}

\paragraph*{(a)}

Euler-Lagrange equations are:

\begin{equation}
\frac{\partial L}{\partial A_\nu} -
  \partial_\mu \left( \frac{\partial L}{\partial \left( \partial_\mu A_\nu \right) } \right) = 0
\end{equation}

Thus:
\begin{equation}
\partial_\mu F^{\mu\nu} = 0
\end{equation}

Putting $E^i = -F^{0i}$ and $\epsilon^{ijk} B^k = -F^{ij}$:

\begin{equation}
0 = \partial_\mu F^{\mu0} = \partial_i E^i = \mathbf{\nabla} \cdot \mathbf{E}
\end{equation}

\begin{equation}
0 = - \partial_\mu F^{\mu j} = - \partial_0 F^{0j} + \partial_i F^{ij} =
\frac{\partial}{\partial t} \mathbf{E}^j - \left(\mathbf{\nabla} \times \mathbf{B} \right)^j
\end{equation}

Putting it all together:

\begin{equation}
\mathbf{\nabla} \cdot \mathbf{E} = 0
\end{equation}

\begin{equation}
\mathbf{\nabla} \times \mathbf{B} = \frac{\partial}{\partial t} \mathbf{E}
\end{equation}

\pagebreak

\paragraph*{(b)}

From (2.17):

\begin{equation}
T^\mu_\nu \equiv
  \frac{\partial {\cal{L}}}{\partial \left( \partial_\mu \phi \right) } \partial_\nu \phi -
  {\cal{L}} \delta^\mu_\nu
\end{equation}

Now putting our Lagrangian to use:

\begin{equation}
\frac{\partial {\cal{L}}}{\partial \left( \partial_\mu A_\nu \right) } =
  -F^{\mu\nu}
\end{equation}

Non-symmetric term of the tensor is:

\begin{equation}
-F^{\mu \rho} \partial_\nu A_\rho =
  -F^{\mu \rho} F_{\nu \rho} - F^{\mu \rho} \partial_\rho A_\nu 
\end{equation}

Symmetrize:

\begin{equation}
\partial_\lambda K^{\lambda \mu \nu} =
  \partial_\lambda \left( F^{\mu \lambda} A^\nu \right) =
  F^{\mu \lambda} \partial_\lambda A^\nu
\end{equation}

\begin{equation}
\hat{T}^{\mu \nu} = T^{\mu\nu} + \partial_\lambda K^{\lambda \mu \nu}
  = - g_{\rho \sigma} F^{\mu \rho} F^{\nu \sigma} -
    {\cal{L}} g^{\mu\nu}
\end{equation}

Which is obviously symmetric now! Working out some particular
components:

\begin{equation}
\hat{T}^{00} = -g_{\rho \sigma} F^{0\rho} F^{0\sigma} - {\cal{L}} =
  F^{0i} F^{0i} - {\cal{L}} = |\mathbf{E}|^2 - {\cal{L}}
\end{equation}

Let's see what ${\cal{L}}$ is:

\begin{equation}
{\cal{L}} = -\frac{1}{4} F^{\mu\nu} F_{\mu\nu} =
  +\frac{1}{2} F^{0i} F^{0i} - \frac{1}{2} F^{ij} F^{ij} =
  \frac{1}{2} \left( |\mathbf{E}|^2 - |\mathbf{B}|^2 \right)
\end{equation}

Finally:

\begin{equation}
\hat{T}^{00} =
  \frac{1}{2} \left( |\mathbf{E}|^2 + |\mathbf{B}|^2 \right)
\end{equation}

Now about spatial components:

\begin{equation}
\hat{T}^{0i} = -g_{\rho \sigma} F^{0 \rho} F^{i \sigma} =
  F^{0 j} F^{i j} = E^j \epsilon_{ijk} B^k =
  \mathbf{E} \times \mathbf{B}
\end{equation}

\pagebreak

\section*{2.2}

\begin{equation}
S = \int d^4 x \left(
  \partial_\mu \phi^* \partial^\mu \phi - m^2 \phi^* \phi \right)
\end{equation}

\paragraph*{(a)}

Conjugate momenta to $\phi(x)$ and $\phi^*(x)$ are of course:

\begin{equation}
\pi(x) =
  \frac{\partial {\cal{L}}}
    {\partial \left( \partial_0{\phi(x)} \right)} =
  \partial^0 \phi^*(x)
\end{equation}

\begin{equation}
\pi^*(x) =
  \frac{\partial {\cal{L}}}
    {\partial \left( \partial_0{\phi^*(x)} \right)} =
  \partial^0 \phi(x)
\end{equation}

Canonical commutation relations:

\begin{equation}
[ \phi(x), \pi(y) ] \equiv i \delta^4 (x - y)
\end{equation}

\begin{equation}
[ \phi^*(x), \pi^*(y) ] \equiv i \delta^4 (x - y)
\end{equation}

\textit{(All other commutators between $\phi$, $\phi*$, $\pi$, and
$\pi*$ are zero). }

\

Hamiltonian is:

\begin{equation}
\begin{split}
H & =
  \int d^3x \; \pi \partial_0 \phi + \pi^* \partial_0 \phi^8 -
    {\cal{L}} \\
  & =
  \int d^3x \; 2 \pi \pi^* - \partial_0 \phi^* \partial^0 \phi +
    \partial_i \phi^* \partial^i \phi + m^2 \phi^* \phi \\
  & = \int d^3x \; \pi \pi^* +
    \mathbf{\nabla} \phi^* \cdot \mathbf{\nabla} \phi +
    m^2 \phi^* \phi
\end{split}
\end{equation}

Heisenberg equation of motion:

\begin{equation}
\frac{\partial \phi}{\partial t} = i [ H, \phi ] =
  \pi^*
\end{equation}

\begin{equation}
\begin{split}
\frac{\partial^2 \phi}{\partial t^2} & =
  i[ H, \frac{\partial \phi}{\partial t} ] =
  i[ H, \pi^*] \\
  & = i \left[
    \int d^3x \;
      \mathbf{\nabla} \phi^*(x) \cdot \mathbf{\nabla} \phi(x) +
      m^2 \phi^*(x) \phi(x),
    \pi^*(y) \right] \\
  & = i \left[
    \int d^3x \;
      -\phi^*(x) \cdot \mathbf{\nabla}^2 \phi(x),
    \pi^*(y) \right] - m^2 \phi(y) \\
  & = + \mathbf{\nabla}^2 \phi(y) - m^2 \phi(y)
\end{split}
\end{equation}

Finally we get:

\begin{equation}
\left( \frac{\partial^2}{\partial t^2} - \mathbf{\nabla}^2 +
  m^2 \right) \phi = 0
\end{equation}

Which is of course is the Klein-Gordon equation.

\paragraph*{(b)}

Introduce creation and annihilation operators expansion for the
fields:

\begin{equation}
\phi(x) = \int \frac{d^3p}{(2 \pi)^3}
  \frac{1}{ \sqrt{ 2 E_\mathbf{p} } }
  \left( a_\mathbf{p} e^{-i p \cdot x} +
         b^\dagger_\mathbf{p} e^{i p \cdot x}
         \right) \bigg\rvert_{p^0 = E_\mathbf{p}}
\end{equation}

\begin{equation}
\phi^*(x) = \int \frac{d^3p}{(2 \pi)^3}
  \frac{1}{ \sqrt{ 2 E_\mathbf{p} } }
  \left( b_\mathbf{p} e^{-i p \cdot x} +
         a^\dagger_\mathbf{p} e^{i p \cdot x}
         \right) \bigg\rvert_{p^0 = E_\mathbf{p}}
\end{equation}

\begin{equation}
\pi^*(x) = \frac{\partial}{\partial{t}} \phi(x) ; \;
\pi(x) = \frac{\partial}{\partial{t}} \phi^*(x)
\end{equation}

With usual commutation relations:

\begin{equation}
[ a_\mathbf{p}, a^\dagger_\mathbf{p'} ] =
  [ b_\mathbf{p}, b^\dagger_\mathbf{p'} ] =
  (2 \pi)^3 \delta^3(\mathbf{p} - \mathbf{p}')
\end{equation}

\textit{(The rest of commutators are zero).}

\

Let's check canonical commutation relations between momenta and
the fields (skipping the resolution of the momentum delta function,
and assuming $x^0 = y^0$):

\begin{equation}
\begin{split}
[ \phi(\mathbf{x}), \pi(\mathbf{y}) ] &
  = \int \frac{d^3p}{(2\pi)^3}
    \frac{1}{2 E_p} i E_p \left(
    e^{ip \cdot (x - y)} + e^{ip \cdot (y - x)} \right) \\
  & = i \delta^3 (\mathbf{x} - \mathbf{y})
\end{split}
\end{equation}

\textit{(Similar result holds for $\phi^*$ and $\pi*$ too)}.

\paragraph*{(c)} Skipping few infinities and canceling the terms
where $p' = -p$:

\begin{equation}
\begin{split}
Q & = \int d^3x \frac{i}{2} \left( \phi^* \pi^* - \pi \phi \right) \\
  & = \int \frac{d^3p}{(2 \pi)^3}
    \frac{1}{2}
    \left(
    	-b^\dagger_\mathbf{p} b_\mathbf{p} +
    	a^\dagger_\mathbf{p} a_\mathbf{p}
    \right)
\end{split}
\end{equation}

The charges are $\frac{1}{2}$ and $-\frac{1}{2}$ for $a$ and $b$
respectively.

\pagebreak

\paragraph*{(d)}

To see if the charge $Q^i$ is conserved, let's compute the commutator
with $H$:

\begin{equation}
H = \sum_{a=1,2} \int d^3x \; \pi_a \pi_a^* +
    \mathbf{\nabla} \phi_a^* \cdot \mathbf{\nabla} \phi_a +
    m^2 \phi_a^* \phi_a
\end{equation}
    
\begin{equation}
Q^i = \int d^3x \frac{i}{2} \left(
  \phi_a^* \left( \sigma^i \right)_{ab} \pi_b^* -
  \pi_a \left( \sigma^i \right)_{ab} \phi_b
  \right)
\end{equation}

\begin{equation}
\frac{\partial}{\partial t} Q^i = i [ H, Q^i ]
\end{equation}

Now for commutators:

\begin{equation}
\int
  [ \pi_c \pi^*_c, \; \phi^*_a \left(\sigma^i\right)_{ab} \pi^*_b ] =
  \delta^{ac} \pi_c \left(\sigma^i\right)_{ab} \pi^*_b =
  -i \pi_a \left(\sigma^i\right)_{ab} \pi^*_b
\end{equation}

\begin{equation}
\int
  [ \pi_c \pi^*_c, \; \pi_a \left(\sigma^i\right)_{ab} \phi_b ] =
  -i \pi_a \left(\sigma^i\right)_{ab} \pi^*_b
\end{equation}

These are precisely the same and thus cancel each other. Next:

\begin{equation}
[ \nabla \phi^*_c \cdot \nabla \phi_c,
  \; \phi^*_a \left(\sigma^i\right)_{ab} \pi^*_b] =
  -i \phi^*_a \left(\sigma^i\right)_{ab} \nabla^2 \phi_b
\end{equation}

\textit{(...and the same term for conjugated part of Q)}.

\

Finally:

\begin{equation}
[ m^2 \phi^*_c \phi_c,
  \; \phi^*_a \left(\sigma^i\right)_{ab} \pi^*_b ] =
  i m^2 \phi^*_a \left(\sigma^i\right)_{ab} \phi_b
\end{equation}
  
\textit{(...and the same term for conjugated part of Q)}.

\

Everything cancels out perfectly and we get:

\begin{equation}
\frac{\partial}{\partial t} Q^i = 0
\end{equation}

Some intermediate steps before working out commutators of charges:

\begin{equation}
\begin{split}
& [ \phi^*_a \left(\sigma^i\right)_{ab} \pi^*_b,
  \phi^*_c \left(\sigma^j\right)_{cd} \pi^*_d ] = \\
& \; \; \phi^*_a \left(\sigma^i\right)_{ab} \pi^*_b
    \phi^*_c \left(\sigma^j\right)_{cd} \pi^*_d -
  \phi^*_c \left(\sigma^j\right)_{cd} \pi^*_d
    \phi^*_a \left(\sigma^i\right)_{ab} \pi^*_b = \\
& \; \; \phi^*_a \left(\sigma^i\right)_{ab} \pi^*_b
    \phi^*_c \left(\sigma^j\right)_{cd} \pi^*_d +
    i \phi^*_c \left(\sigma^j \sigma^i\right)_{cb} \pi^*_b - \\
& \; \; \; \; \; \; \; \;
    i \phi^*_a \left(\sigma^j\right)_{cd} \phi^*_c \pi^*_b
      \left(\sigma^i\right)_{ab} \pi^*_d = \\
& \; \;
   \phi^*_c \left([ \sigma^j , \sigma^i ]\right)_{cb} \pi^*_b =
   2 \epsilon_{ijk} \phi^*_a \left(\sigma^k\right)_{ab} \phi^*_b
\end{split}
\end{equation}

The other term is a conjugate of this. Thus we get:

\begin{equation}
[ Q^i, Q^j] = 2 \epsilon_{ijk} Q^k
\end{equation}

Which means that the commutation relations of $Q^i$ are the same as of
generators of Lie Algebra $su(2)$!

\end{document}